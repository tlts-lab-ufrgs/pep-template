%Template para TCC CIC UFRGS
%Author: Lucas Schnorr (Adaptado para overleaf por Rubens dos Santos)

% exemplo genérico de uso da classe iiufrgs.cls
% $Id: iiufrgs.tex,v 1.1.1.1 2005/01/18 23:54:42 avila Exp $
% 
% This is an example file and is hereby explicitly put in the
% public domain.
% 
\documentclass[ppgc,tc,english]{iiufrgs}
% Para usar o modelo, deve-se informar o programa e o tipo de documento.
% Programas :
% * ppgc      -- Programa de Pós Graduação em Computação
% 
% Tipos de Documento:
% * tc                -- Trabalhos de Conclusão (apenas cic e ecp)
% 
% Outras Opções:
% * english    -- para textos em inglês
% * openright  -- Força início de capítulos em páginas ímpares (padrão da
% biblioteca)
% * oneside    -- Desliga frente-e-verso
% * nominatalocal -- Lê os dados da nominata do arquivo nominatalocal.def


% Use unicode
\usepackage[utf8]{inputenc}   % pacote para acentuação

% Necessário para incluir figuras
\usepackage{graphicx}         % pacote para importar figuras

\usepackage{times}            % pacote para usar fonte Adobe Times
% \usepackage{palatino}
% \usepackage{mathptmx}       % p/ usar fonte Adobe Times nas fórmulas

\usepackage[alf,abnt-emphasize=bf]{abntex2cite}	% pacote para usar citações abnt

\usepackage{xcolor}

\newcommand{\tlts}[1]{{\color{red}#1}}
\newcommand{\aluno}[1]{{\color{blue}#1}}
\newcommand{\todo}[1]{{\color{orange}#1}}


% 
% Informações gerais
% 
\title{Title}

\author{Surname}{Name}

\advisor[Prof.~Dr.]{da Silveira}{Thiago Lopes Trugillo}

% 
\keyword{First keyword}
\keyword{Second keyword}
\keyword{Third keyword}
\keyword{Fourth keyword}

% 
\begin{document} 



\maketitle

\begin{abstract}
Abstract.
    \tlts{Thiago}
    \aluno{Aluno}
    \todo{TODO}

\end{abstract}
\tableofcontents
\chapter{Introduction}


\chapter{Related Work}


\chapter{Proposed Methodology}

\section{Proposed Schedule}

\begin{table}[h]
    \caption{Proposed schedule}
    % OBS: não use \begin{center}, pois este aumenta o espaçamento entre a caption/legend e a tabela
    % Para figuras, a aparência é melhor com o espaçamento extra
    \centering
        \begin{tabular}{c|c|c|c|c|c|c|c}
          \hline
          {Activity}  &   \textit{X}  &   \textit{Y}  &   \textit{Z}  &   \textit{W}  &   \textit{X}  &   \textit{Y}  &   \textit{Z}\\
          \hline
                  & $\times$& & & & & &   \\
                  & $\times$& & & & & &   \\
                  & $\times$& & & & & &   \\
          \hline
        \end{tabular}
     \label{tbl:ex1}
\end{table}


\bibliographystyle{abntex2-alf}
\bibliography{biblio}

\end{document}
